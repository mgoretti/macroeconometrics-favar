\begin{table}[H] 
 \centering 
\begin{threeparttable} 
\caption{\textsc{Information Criteria}} \label{tab:IC} 
\begin{tabular*}{0.55 \textwidth }{lcccccc} 
 \toprule \toprule  
 lag & AIC & BIC & HQIC & FPE & CVabs & Cvrel \tabularnewline \midrule 1  &    -2.61 &    -2.23 &    -2.46 &     0.08 &    29.73 &     4.77 \\ 
 2  &    -2.89 &    -2.12 &    -2.58 &     0.06 &    29.11 &     4.67 \\ 
 3  &    -3.06 &    -1.91 &    -2.59 &     0.05 &    29.32 &     4.69 \\ 
 4  &    -3.10 &    -1.57 &    -2.48 &     0.05 &    29.80 &     4.77 \\ 
 5  &    -3.19 &    -1.27 &    -2.42 &     0.04 &    30.20 &     4.83 \\ 
 6  &    -3.16 &    -0.85 &    -2.23 &     0.04 &    30.62 &     4.91 \\ 
 7  &    -3.19 &    -0.50 &    -2.10 &     0.04 &    31.14 &     4.99 \\ 
 8  &    -3.25 &    -0.17 &    -2.01 &     0.04 &    31.30 &     5.00 \\ 
 9  &    -3.23 &     0.23 &    -1.84 &     0.04 &    32.22 &     5.15 \\ 
 10 &    -3.22 &     0.62 &    -1.67 &     0.04 &    33.02 &     5.28 \\ 
 11 &    -3.19 &     1.03 &    -1.49 &     0.05 &    33.69 &     5.39 \\ 
 12 &    -3.06 &     1.55 &    -1.20 &     0.05 &    34.74 &     5.56 \\ 
 \bottomrule \bottomrule 
 \end{tabular*} 
\begin{tablenotes} 
\small 
\item \emph{ \footnotesize{ Notes : This table reports the final prediction error (FPE), Akaike's information criterion (AIC), Swarz's Bayesian information criterion (BIC), the Hannan and Quinn information criterion (HQIC), Cross validation criterion (absolute CVabs and relative CVrel) lag-order selection statistics for a series of vector autoregressions of order 1 through a maximum of 12 lags.  } } 
\end{tablenotes} 
\end{threeparttable} 
\end{table} 
